\documentclass[11pt]{article}
%justify text
\usepackage{ragged2e}
%set page margin size
\usepackage[left=1.5cm, right=1.5cm, top=1.5cm]{geometry}

\usepackage[spanish]{babel}

\newcommand{\lecturetitle}[1] {
    \textbf{Estructuras de datos y algoritmos}
    \newline
    \textbf{Clase #1}
    \newline
    \textbf{Profesor Yerko Ortiz}
    \newline
    \textbf{Universidad Diego Portales}
    \newline
    \noindent\rule{\textwidth}{1pt}
}

\newcommand{\lecturesection}[1] {
    \section{
        \textbf{
            #1
        }
    }
}

\newcommand{\lectureobjective}[1]{
    \underline{Objetivo:} #1
}


\begin{document}

%\lecturetitle{1}
\title{Estructuras de Datos y Algoritmos: Notación Asintótica}
\date{}
\author{Yerko Ortiz}

\maketitle
\flushleft
\justifying
\lectureobjective{
    El objetivo de la segunda clase es estudiar una serie de funciones matemáticas bien conocidas y definir la notación Big-oh, Theta y Omega.
}

\tableofcontents

\lecturesection{Introducción}
Para poder medir la eficiencia de un algoritmo y así mismo poder comparar qué algoritmo podría ser mejor para un determinado problema, 
es necesario tener un método que describa la eficiencia del algoritmo en términos del tamaño de entrada del problema.

\lecturesection{Notación Asintótica}
La notación asintótica describe el comportamiento de una función matemática. En el contexto del curso, la notación asintótica caracteriza el tiempo de ejecución o espacio de memoria de un algoritmo mediante una función matemática. 

Las ventajas de describir la eficiencia de un algoritmo con notación asintótica son:
\begin{itemize}
    \item Describe la eficiencia en función de un tamaño variable de entrada.
    \item Caracteriza el uso de recursos del algoritmo mediante una función matemática sin constantes ni términos de menor orden.
    \item Es independiente de las especificaciones de la maquina que ejecuta el algoritmo.
\end{itemize}
\newpage
\lecturesection{Crecimiento de funciones}
Puesto que el tamaño de entrada un algoritmo es arbitrario, el tiempo de ejecución o espacio de memoria de un 
algoritmo puede ser caracterizado mediante una función matemática que considere el tamaño de entrada.

\begin{table}[!ht]
    \centering
    \begin{tabular}{|l|l|l|l|l|l|}
    \hline
        Input & Constante & Logaritmo & Lineal & Linearítmica & Cuadrática \\ \hline
        $f(N)$ & $1$ & $lgN$ & $N$ & $NlgN$ & $N^2$ \\ \hline
        f(1000) & 1 & 10 & 1000 & 10000 & 1000000 \\ \hline
        Tiempo & 1 ns & 10 ns & 1 $\mu$ s & 10 $\mu$ s & 1 ms \\ \hline
        Memoria & 1 B & 10 B & 1KB & 10KB & 10MB \\ \hline
    \end{tabular}
\end{table}

\lecturesection{Notación Big O}
$\mathcal{O}(g(N))$ = $\{ f(N):$ existen constantes positivas $c$ y $N_0$ de forma que $0 \leq f(N) \leq cg(N)$, para todo $N \ge N_0 \}$
\begin{itemize}
    \item Big O caracteriza el peor caso de un algoritmo.
\end{itemize}
\lecturesection{Notación Theta}
$\Theta(g(N))$ = $\{ f(N):$ existen constantes positivas $c_1$, $c_2$ y $N_0$ de forma que $0 \leq c_{1}g(N)\leq f(N) \leq c_{2}g(N)$, para todo $N \ge N_0 \}$
\begin{itemize}
    \item Theta caracteriza el caso promedio de un algoritmo.
\end{itemize}
\lecturesection{Notación Omega}
$\Omega(g(N))$ = $\{ f(N):$ existen constantes positivas $c$ y $N_0$ de forma que $0 \leq cg(N) \leq f(N)$, para todo $N \ge N_0 \}$
\begin{itemize}
    \item Omega caracteriza el mejor caso de un algoritmo.
\end{itemize}


\end{document}
