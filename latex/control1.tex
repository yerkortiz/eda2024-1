\documentclass{exam}
%assign letters or other incrementing expressions to enumerate
\usepackage[shortlabels]{enumitem}

\firstpageheader{
    Control 1: Análisis asintótico \\
    Profesor Yerko Ortiz \\ 
    Tiempo: 45 minutos
    }{}
    {Nombre: \underline{\hspace{2.5in}} }

\usepackage{listings}
\usepackage{xcolor}

\definecolor{codegreen}{rgb}{0,0.6,0}
\definecolor{codegray}{rgb}{0.5,0.5,0.5}
\definecolor{codepurple}{rgb}{0.58,0,0.82}
\definecolor{backcolour}{rgb}{0.95,0.95,0.92}

\lstdefinestyle{javalisting}{
    backgroundcolor=\color{backcolour},   
    commentstyle=\color{codegreen},
    keywordstyle=\color{magenta},
    numberstyle=\tiny\color{codegray},
    stringstyle=\color{codepurple},
    basicstyle=\ttfamily\normalsize,
    breakatwhitespace=false,         
    breaklines=true,                 
    captionpos=b,                    
    keepspaces=true,                 
    numbers=left,                    
    numbersep=5pt,                  
    showspaces=false,                
    showstringspaces=false,
    showtabs=false,                  
    tabsize=2
}

\lstset{style=javalisting}

\begin{document}
\section{Teoría}
\subsection{Tiempo de ejecución}
Caracterice el tiempo de ejecución de las siguientes expresiones utilizando notación $\mathcal{O}(f(n))$. [5 puntos cada respuesta correcta]
\begin{enumerate}[a.]
    \item $T(N) = N^2 + N\sqrt{N} + 5$
    \vspace{1cm}
    \item $T(N, K) = N \log N + N ^ K$
    \vspace{1cm}
    \item $T(N) = \sqrt{N} + \log N$
    \vspace{1cm}
    \item $T(N, M) = 5N^4 + 2N^2 + M + 1$
    \vspace{1cm}
\end{enumerate}
\subsection{Verdadero y falso}
Para cada una de las siguientes afirmaciones, denote su veracidad con la letra V o su falsedad con la letra F. Para las respuestas falsas justifique su respuesta. [5 puntos cada respuesta correcta]
\begin{enumerate}[a.]
    \item \underline{\hspace{0.7cm}} La notación $\Omega(f(n))$ es utilizada para caracterizar el caso promedio de ejecución de un algoritmo.
    \vspace{1cm}
    \item \underline{\hspace{0.7cm}} Una ventaja del análisis asintótico es que es dependiente de las características de la maquina en la que el algoritmo es ejecutado, es decir, según la velocidad de la cpu y cantidad de memoria el resultado del análisis será distinto.
    \vspace{1cm}
    \item \underline{\hspace{0.7cm}} La eficiencia de un algoritmo caracteriza el uso de algún recurso computacional(cpu, memoria) respecto el tamaño de entrada del algoritmo.
    \vspace{1cm}
    \item \underline{\hspace{0.7cm}} El siguiente algoritmo calcula el logaritmo base dos de un número entero N:
        \begin{lstlisting}[language=java]
        static int f(int n) {
            int l = -1;
            while(n >  0) {
                l++;
                n = n/10;
            }
            return l;
        }
        \end{lstlisting}
\end{enumerate}

\newpage 
\section{Análisis}
Para el siguiente algoritmo[5 puntos cada respuesta correcta]:
\begin{enumerate}[a.]
    \item Describa el algoritmo en términos de entrada/salida, es decir, las propiedades que definen al conjunto de entrada(tipo de datos), como también las propiedades y restricciones que permiten validar si un valor de salida es correcto o no.
    \item Calcule la salida del algoritmo si la entrada es: $\{ n = 2\}$
    \item Calcule la salida del algoritmo si la entrada es: $\{ n = 5\}$
    \item Describa el tiempo de ejecución utilizando notación $\mathcal{O}(f(n))$
\end{enumerate}
\begin{lstlisting}[language=java]
static int f(int n) {
    int s = 0;
    for (int i = 1; i <= n*n; i++) {
        for (int j = 1; j <= i; j++) {
            s++;
        }
    }
    return s;
}
\end{lstlisting}

\end{document}
